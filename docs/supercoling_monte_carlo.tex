\documentclass[a4paper, 11pt]{article}
\usepackage[utf8]{inputenc}
%\usepackage[spanish]{babel}
\usepackage[T1]{fontenc}
\usepackage{fancyhdr} % Paquete de encabezados y pies de pagina
\usepackage{anysize} % Para cambiar los margenes
\usepackage{listings}
\usepackage{xcolor}
\usepackage{graphicx}
\usepackage{caption}
\marginsize{2cm}{2cm}{1cm}{1cm}
\usepackage{amsmath}
\newcommand\norm[1]{\lVert#1\rVert}
\newcommand\normx[1]{\Vert#1\Vert}
%opening
\title{\textbf{Monte Carlo simulations of circular supercoiled polymers}}
\author{César L. Pastrana}
\date{}


\begin{document}

\maketitle

\section{Intersections}
To prevent 

We define two
\begin{equation}
 \vec{r}_1 = \vec{r}_0 + \delta_1\hat{t}_1
\end{equation}
\begin{equation}
 \vec{u}_1 = \vec{u}_0 + \delta_2\hat{t}_2
\end{equation}
The distance between the two lines is then $d(\vec{r}_1, \vec{r}_2) = \norm{\vec{r}_1 - \vec{r}_2}$. We then look for the $\delta_i$ minimizing $d$,
\begin{equation}
\begin{split}
 \frac{\partial d(\vec{r}_1, \vec{r}_2)}{\partial \delta_1} = 0\\
 \frac{\partial d(\vec{r}_1, \vec{r}_2)}{\partial \delta_2} = 0,
 \end{split}
\end{equation}

\section{Knots}
\end{document}
 
